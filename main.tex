% FortySecondsCV LaTeX template
% Copyright © 2019 René Wirnata <rene.wirnata@pandascience.net>
% Licensed under the 3-Clause BSD License. See LICENSE file for details.
%
% Attributions
% ------------
% * fortysecondscv is based on the twentysecondcv class by Carmine Spagnuolo 
%   (cspagnuolo@unisa.it), released under the MIT license and available under
%   https://github.com/spagnuolocarmine/TwentySecondsCurriculumVitae-LaTex
% * further attributions are indicated immediately before corresponding code


%-------------------------------------------------------------------------------
%                             ADDITIONAL PACKAGES
%-------------------------------------------------------------------------------
\documentclass[
  a4paper, 
%   showframes,
%   vline=2.2em,
%   maincolor=cvgreen,
%   sectioncolor=red,
%   subsectioncolor=orange,
%   itemtextcolor=black!80,
%   sidebarwidth=0.4\paperwidth,
%   topbottommargin=0.03\paperheight,
%   leftrightmargin=20pt,
%   proilepicsize=4.5cm,
]{fortysecondscv}

% improve word spacing and hyphenation
\usepackage{microtype}
\usepackage{ragged2e}

% take care of proper font encoding
\ifxetexorluatex
	\usepackage{fontspec}
	\defaultfontfeatures{Ligatures=TeX}
% \newfontfamily\headingfont[Path = fonts/]{segoeuib.ttf} % local font
\else
	\usepackage[utf8]{inputenc}
	\usepackage[T1]{fontenc}
% \usepackage[sfdefault]{noto} % use noto google font
\fi

% enable mathematical syntax for some symbols like \varnothing
\usepackage{amssymb}

% bubble diagram configuration
\usepackage{smartdiagram}
\smartdiagramset{
  % defaut font size is \large, so adjust to harmonize with sidebar layout
  bubble center node font = \footnotesize,
  bubble node font = \footnotesize,
  % default: 4cm/2.5cm; make minimum diameter relative to sidebar size
  bubble center node size = 0.4\sidebartextwidth,
  bubble node size = 0.25\sidebartextwidth,
  distance center/other bubbles = 1.5em,
  % set center bubble color
  bubble center node color = maincolor!70,
  % define the list of colors usable in the diagram
  set color list = {maincolor!10, maincolor!40,
  maincolor!20, maincolor!60, maincolor!35},
  % sets the opacity at which the bubbles are shown
  bubble fill opacity = 0.8,
}

%-------------------------------------------------------------------------------
%                            PERSONAL INFORMATION
%-------------------------------------------------------------------------------
%% mandatory information
% your name
\cvname{\huge{Atıl} \Huge{i}\huge{lerialkan}}
% job title/career
\cvjobtitle{Software Design Engineer,\\[0.2em] \large{Embedded Software Developer}}

%% optional information
% profile picture
\cvprofilepic{pics/218965veskare.png}

% NOTE: ordering in sidebar will mimic the following order
% date of birth
\cvbirthday{January 2, 1991}
% short address/location, use \newline if more than 1 line is required
\cvaddress{Ankara, Turkey}
% phone number
\cvphone{+90 506 422 01 66}
% personal website
% \cvsite{https://pandascience.net}
% email address
\cvmail{atililerialkan@gmail.com}
% Linkedin user account
\cvlinkedin{atililerialkan}
% Github user account
\cvgithub{atilileri}
% pgp key
% \cvkey{4096R/FF00FF00}{0xAABBCCDDFF00FF00}
% any other custom entry
%\cvcustomdata{\faFlag}{Turkey}

%-------------------------------------------------------------------------------
%                              SIDEBAR 1st PAGE
%-------------------------------------------------------------------------------
% add more profile sections to sidebar on first page
\addtofrontsidebar{
	\profilesection{\Large{About Me}}
	\aboutme{
    In my 8 year software engineering career, I have taken part in 5 different aerospace and defense programs. I contributed to software development activities in all life-cycle phases, according to DO-178C. I am quite familiar with:\\
    - \textbf{Avionics}
     \textit{(e.g. IFF Transponder, CVDR)},\\
    - High and low-level software development in \textbf{Safety Critical Systems}
     \textit{(e.g. Bootloader, BSP\&IO Drivers, Application Level Software)},\\
    - \textbf{Graphical User Interface} development
     \textit{(e.g. Qt, Scade Suite, VAPS XT)},\\
    - \textbf{Software Development Life-cycle Methodologies}
     \textit{(e.g. Agile, Waterfall, V-Model)}.\\ \\
    In my master’s program, I was highly involved with \textbf{Machine Learning} and \textbf{Deep Learning} applications on multimedia data \textit{(e.g. Video, Image, Sound)}. My thesis is focused on the \textbf{classification of sound signals}, which can have a great extent in smart cockpit applications.\\ \\
    I have always been an active person, pushing my boundaries to learn more every day. I believe working and solving problems in an international company would be the best next step and a unique opportunity for my career. Sharing a workspace with people from different cultures and disciplines that have the same interests is a great chance to develop myself both as a person and as an engineer.\\
    }
}

%-------------------------------------------------------------------------------
%                              SIDEBAR 2nd PAGE
%-------------------------------------------------------------------------------
\addtobacksidebar{

	\profilesection{\Large{Programming Languages}}
	\pointskill{\faCaretRight}{\textcolor{black}{C}}{4}
	\pointskill{\faCaretRight}{\textcolor{black}{C++}}{4}
	\pointskill{\faCaretRight}{\textcolor{black}{Python}}{3}
	\pointskill{\faCaretRight}{\textcolor{black}{C\#}}{2}
	\pointskill{\faCaretRight}{\textcolor{black}{Java}}{1}
	\pointskill{\faCaretRight}{\textcolor{black}{Assembly}}{1}
	
    \profilesection{\Large{Skills \& Knowledge}}
        \skilltagpri{ARINC 429}
        \skilltagpri{ARINC 653}
        \skilltagpri{MIL-STD-1553}
        \skilltagpri{ARINC 661}
        \skilltagpri{DO-178}
        \skilltagsec{DO-248}
        \skilltagpri{DO-330}
        \skilltagsec{DO-331}
        \skilltagsec{DO-332}
        \skilltagsec{DO-297}
        \skilltagsec{AADL}
        \skilltagsec{SysML}
        \skilltagsec{FACE}
        \skilltagsec{MOSA}
        \skilltagsec{Open Architecture}
        \skilltagsec{IMA}
        \skilltagpri{U-Boot}
        \skilltagpri{Serial}
        \skilltagsec{RS-232}
        \skilltagsec{RS-422}
        \skilltagsec{RS-485}
        \skilltagpri{I2C}
        \skilltagpri{USB}
        \skilltagpri{Ethernet}
        \skilltagsec{PCI}
        \skilltagsec{SPI}
        \skilltagsec{CAN}
        \skilltagsec{UART}
        \skilltagpri{Flash}
        \skilltagpri{NVRAM}
        \skilltagsec{OpenGL}
        \skilltagpri{Integrity tuMP}
        \skilltagpri{RTOS}
        \skilltagpri{AdaMULTI}
        \skilltagpri{Enterprise Architect}
        \skilltagpri{VAPS XT}
        \skilltagpri{Scade Suite}
        \skilltagpri{Qt Framework}
        \skilltagsec{.NET Framework}
        \skilltagpri{Jira}
        \skilltagpri{Confluence}
        \skilltagsec{Crucible}
        \skilltagsec{Jenkins}
        \skilltagpri{DOORS}
        \skilltagpri{Svn}
        \skilltagpri{Git}
        \skilltagpri{Agile}
        \skilltagpri{Waterfall}
        \skilltagpri{V-Model}
	% include gosquare national flags from https://github.com/gosquared/flags;
	% naming according to ISO 3166-1 alpha-2 country codes
	\graphicspath{{pics/flags/}}

	\profilesection{\Large{Foreign Languages}}
	\pointskill{\flag{GB.png}}{English}{4}


	\profilesection{\Large{Hobbies}}
        \textcolor{black}{I was a professional \textit{American Football} player for the last decade and a professional \textit{Basketball} player for 5 years. I believe taking responsibility in those associations and wearing the national team uniform was a huge gain for both my professional life and communication skills.\\
        }
        \\
	    \skill{\faMedal}{Sports}
    	    \skill[1em]{\faFootballBall}{American Football\\Turkish National Team Player}
    	    \skill[1em]{\faBasketballBall}{Basketball\\Licensed College Player}
	    \skill{\faTv}{Movies, Tv Shows}
}


%-------------------------------------------------------------------------------
%                         TABLE ENTRIES RIGHT COLUMN
%-------------------------------------------------------------------------------
\begin{document}

\makefrontsidebar

\cvsection{Working Experience}
\begin{cvtable}[1.5]
    \cvitem{2021-current}{IMODE Team Lead}{\href{https://www.tusas.com/en/}{Turkish Aerospace}}{
    	IMODE, Kit Development\\
    	Project covers maintenance of the current tool and developing additional features, including following key modules:
        \begin{itemize}
            \item{ARINC 661 Toolkit}
    	    \item{Architecture Design Toolkit (AADL)}
    	    \item{Tool Qualification Kit}
    	    \item{Software Development Kit}
        \end{itemize}
        Roles and responsibilities in the project:
        \begin{itemize}
            \item{Scrum Master, Sprint Planning}
    	    \item{Researching new technologies and standards in the project path, identifying challenging topics and risks}
    	    \item{Co-instructor at \textit{Model-Based Design} lecture at \textit{Yildiz Technical University Avionics Engineering} master's programme}
        \end{itemize}
    }
	\cvitem{2019 -- 2021}{Senior Software Design Engineer}{\href{https://www.tusas.com/en/}{Turkish Aerospace}}{
    IMODE: Integrated Model-Based Design \& Development Environment, Tool Development.\\
    IMODE is a tool for model-based design and development, which is developed by Turkish Aerospace, as a competition to VAPS and SCADE (Suite). 
    \begin{itemize}
        \item{Development of plans, standards, design, requirements and source code}
        \item{Participated modules:
        \begin{itemize}
            \item{Project \& Model Management}
            \begin{itemize}
                \item{Save \& Load (XML, JSON)}
                \item{Referencing}
            \end{itemize}
            \item{Template-Based Code Generation}
            \begin{itemize}
                \item{Implementation of a Code Generator for C Language (based on the template)}
            \end{itemize}
            \item{License Management (Public Key Authentication)}
            \begin{itemize}
                \item{Node-locked}
                \item{Floating}
            \end{itemize}
            \item{Graphical and logical model designs for \textit{Advanced Jet Trainer and Light Attack Aircraft, PVI (Pilot Vehicle Interface) Development} project}
        \end{itemize}
        }
        %\item{Responsibilities as a Team Lead:}
        %\item{Development of plans, standards, design, requirements and source code for modules: Project \& Model Management (save/load/parsing/referencing), Graphical and Logical Model Architecture, Code Generation (including template architecture). c template ornek olarak\\
        \item{Implemented in C++ with Qt Framework.}
    \end{itemize}
    }
	\cvitem{2017 -- 2019}{Software Design Engineer}{\href{https://www.aselsan.com.tr/en}{Aselsan} (on behalf of \href{https://www.tusas.com/en/}{Turkish Aerospace})}{
		New Generation Trainer Aircraft, OFP Development
		\begin{itemize}
		    \item{Development of software requirements and source code of the following modules:
            \begin{itemize}
                \item{IFF (Identification Friend or Foe) Transponder}
                \item{CVRFDR (Cockpit Voice (CVR) and Flight Data (FDR) Recorder)}
                \item{Initial development of some radar pages using VAPS}
            \end{itemize}}
            \item{Implemented in C++ using IBM Rational Rhapsody.}
            \item{Supported verification activities.}
		\end{itemize}
	}
\end{cvtable}

\newpage
\makebacksidebar

\begin{cvtable}[1.5]
	\cvitem{2016 -- 2017}{Software Design Engineer}{\href{https://www.tusas.com/en/}{Turkish Aerospace}}{
	    New Generation Trainer Aircraft, LRU Firmware
        \begin{itemize}
            \item{Development of source code for modules: 
            \begin{itemize}
                \item{Bootloader}
                \item{I2C Device Driver}
            \end{itemize}
            for Symbol Generator (SG) and Rudder Assistant System (RAS) hardwares.}
            \item{Source code implemented in C.}
        \end{itemize}
		Military Transport Aircraft, Avionics Modernization
        \begin{itemize}
            \item{Development of an ARINC 429 data simulator tool for the verification environment using .NET Framework in C\#.}
        \end{itemize}
	}
    \cvitem{2014 -- 2016}{Software Engineer}{\href{https://www.ayesas.com/en}{AYESAŞ}}{
    	New Generation Trainer Aircraft, LRU Firmware
        \begin{itemize}
            \item{Development of requirements, for the firmwares of the following LRUs (Line Replaceable Unit) using IBM Rational DOORS:
            \begin{itemize}
                \item{Intercommunication System (ICS)}
                \item{Multi-Function Display (MFD)}
                \item{Keyboard/Cockpit Display Unit (KDU/CDU)}
            \end{itemize}}
        \end{itemize}
        MALE class Unmanned Aerial Vehicle, BSP \& IO Drivers
        \begin{itemize}
            \item{Development of plans, standards, design, requirements and source code for modules: 
            \begin{itemize}
                \item{Bootloader (setting up Minimum Processing Environment)}
                \item{Dataloader (download \& flash application images)}
                \item{BSP \& IO Device Drivers (Ethernet, Serial, Flash, NVRAM)}
            \end{itemize}}
            \item{Source code implemented in C and Assembly.}
            \item{Also participated in verificiation activities.}
        \end{itemize}
    }
	\cvitem{2013 -- 2014}{Software Engineer}{\href{https://www.bilisim.com.tr}{Bilişim AŞ}}{
        DGCA Information Administration System
        \begin{itemize}
            \item{Development, maintenance and support using Java Server Faces.}
        \end{itemize}
        Business Intelligence Application
        \begin{itemize}
            \item{Development of the R\&D project using PHP, Java and Javascript.}
            \item{Multiple database management systems like OracleDB, MySQL and MsSQL were used.}
        \end{itemize}
    }
\end{cvtable}

\cvsection{Education}
\begin{cvtable}[1.5]
	\cvitem{2016 -- 2019}{MSc. at Multimedia Computing}{Middle East Technical University}
		{Courses: Signal Processing, Machine Learning, Motion Capture, Speech Recognition, Image Processing, Multimedia Standards, Parallel Programming on GPUs, Deep Learning.\\Used Python for development.}
	\cvitem{}{Master Thesis}{}
		{Speaker and Posture Classification using Instantaneous Acoustic Features of Breath Signals}
	\cvitem{2009 -- 2013}{BSc. at Computer Engineering}{Hacettepe University}
		{Focus: Operating Systems, Data Structures and Algorithms, Software Design Patterns}
\end{cvtable}

\cvsection{Publications}
\begin{cvtable}
	\cvpubitem{Speaker and Posture Classification using Instantaneous Intraspeech Breathing Features}{Preprint}
	    {\href{https://arxiv.org/abs/2005.12230}{arXiv:2005.12230}}{2020}
	\cvpubitem{BreathBase: Intra-Speech Breathing Dataset}{Dataset (Open Access)}
	    {\href{https://doi.org/10.5281/zenodo.3841039}{10.5281/zenodo.3841039}}{2020}
\end{cvtable}

\cvsection{Certifications}
\begin{cvtable}
	\cvitem{2017}{Deep Learning Fundamentals}{NVIDIA Deep Learning Institute}{}
	\cvitem{2016}{ALES: 83 (equivalent to GRE: 157, GMAT: 610)}{Exam Score}{}
	\cvitem{2016}{YDS: 95 (equivalent to TOEFL-IBT: 114, IELTS: 7,5)}{Exam Score}{}
	\cvitem{2015}{DO-178C Short Courses}{AFuzion Inc}{}
	\cvitem{2015}{Using DOORS for Requirement Management}{PROYA}{}
	\cvitem{2014}{Integrity BSP \& IO Drivers}{Green Hills Software}{}
\end{cvtable}

% \cvsignature

\end{document} 
